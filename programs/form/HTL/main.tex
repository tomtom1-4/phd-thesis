\documentclass[11pt]{article}

\usepackage{amsmath,amssymb,amsfonts}
\usepackage[margin=1in]{geometry}

\begin{document}

\section*{Integrating Out a Heavy Quark: Explicit Sketch}

\subsection*{1. Setting Up the Problem}

Consider QCD with all quark flavors, including a heavy top quark of mass $m_t$.
In (Euclidean-signature) notation, the relevant part of the action is:
$

  S_{\text{QCD}} \;=\; \int d^4x \,\Bigl[
    \tfrac{1}{4}\,G_{\mu\nu}^a\,G_{\mu\nu}^a
    \;+\;\bar{t}\,\bigl(\gamma^\mu D_\mu + m_t\bigr)\,t
    \;+\;\ldots
  \Bigr],

$
where
$

  G_{\mu\nu}^a \;=\;\partial_\mu A_\nu^a - \partial_\nu A_\mu^a + g_s\,f^{abc}\,A_\mu^b\,A_\nu^c

$
is the non-Abelian field strength tensor (with gauge coupling $g_s$ and structure constants $f^{abc}$), and $D_\mu = \partial_\mu - i\,g_s\,A_\mu^a\,T^a$ is the covariant derivative in the quark representation. Throughout, $\ldots$ indicates terms involving the lighter quarks, ghosts, etc., which for our purposes are not crucial to show explicitly.

The partition function (generating functional) is written schematically as:
$

  Z \;=\; \int \!\mathcal{D}A_\mu\;\mathcal{D}(\bar{t},t)\;\exp\!\bigl[
    -\,S_{\text{QCD}}[A_\mu,\bar{t},t,\ldots]
  \bigr].

$
When we ``integrate out'' the top quark, we perform the path integral over $t,\bar{t}$:
$

  \int \!\mathcal{D}(\bar{t},t)\;\exp\!\bigl[
    -\,\int d^4x\;\bar{t}\,\bigl(\gamma^\mu D_\mu + m_t\bigr)\,t
  \bigr]
  \;=\;
  \det\!\bigl(\gamma^\mu D_\mu + m_t\bigr).

$
Hence the effective action obtains a contribution
$

  S_{\text{eff}}[A_\mu]
  \;=\;
  -\,\ln\,\det\!\bigl(\gamma^\mu D_\mu + m_t\bigr) \;+\; \ldots~,

$
where the ``$\ldots$'' includes the original gauge-field action and other quark terms. We seek the local operator(s) generated from the large-$m_t$ expansion of $\ln\det(\gamma^\mu D_\mu + m_t)$.

\subsection*{2. Factor Out the Large Mass and Expand}

Define the Dirac operator
$

  \mathcal{O}
  \;=\;
  \gamma^\mu D_\mu + m_t.

$
Then
$

  \ln \det(\mathcal{O})
  \;=\;
  \mathrm{Tr}\,\ln(\mathcal{O}).

$
We can write
$

  \mathcal{O}
  \;=\;
  m_t \,\Bigl(1 + X\Bigr),
  \quad\text{where}\quad
  X \;=\;\frac{\gamma^\mu D_\mu}{m_t}.

$
Thus,
$

  \ln \det(\mathcal{O})
  \;=\;
  \mathrm{Tr}\,\ln\!\bigl(m_t\bigr)
  \;+\;
  \mathrm{Tr}\,\ln\!\bigl(1 + X\bigr).

$
The piece $\mathrm{Tr}\,\ln(m_t)$ is field-independent and does not affect the gluon dynamics, so we can drop it. Hence,
$

  \ln \det(\mathcal{O})
  \;=\;
  \mathrm{Tr}\,\ln(1 + X)
  \;=\;
  \sum_{n=1}^{\infty}
  \frac{(-1)^{n+1}}{n}\,\mathrm{Tr}\!\bigl(X^n\bigr),
  \quad
  X \;=\;\frac{\gamma^\mu D_\mu}{m_t}.

$

\subsection*{3. Power-Series Terms and Gauge Invariance}

Each $X^n$ is $(\gamma^\mu D_\mu)^n/m_t^n$. To obtain a local gauge-invariant operator of \emph{dimension four} in the fields, we need to examine terms of a certain total dimension in derivatives and fields. Since $D_\mu$ is of mass dimension 1, dividing by the large mass $m_t$ reduces that dimension by 1 each time.

It turns out that the first nontrivial local operator at dimension four (beyond a constant) is
$

  G_{\mu\nu}^a\,G_{\mu\nu}^a.

$
This arises in part because the field strength $G_{\mu\nu}^a$ appears through commutators of covariant derivatives, $[D_\mu,D_\nu]\propto G_{\mu\nu}^a$, and the gamma matrices $\gamma^\mu$ can produce a factor of $\sigma^{\mu\nu} = \tfrac{1}{2}[\gamma^\mu,\gamma^\nu]$ in the expansion.

\subsection*{4. Schwinger--DeWitt (Heat Kernel) Perspective}

A more systematic method uses the Schwinger--DeWitt expansion in (Euclidean) proper time:
$

  \Gamma[A_\mu]
  ~=~
  -\,\ln\!\det\!\bigl(\gamma^\mu D_\mu + m_t\bigr)
  ~=~
  -\int_{0}^{\infty}
  \frac{ds}{s}\,e^{-\,m_t^2\,s}
  \,\mathrm{Tr}\!\Bigl[e^{-\,s\,(\gamma^\mu D_\mu)^2}\Bigr].

$
One then expands $e^{-s\,(\gamma^\mu D_\mu)^2}$ in a heat-kernel series. The coefficient of $s^0$ (in 4D) turns out to include the term $\propto G_{\mu\nu}^a\,G_{\mu\nu}^a$, so after integrating over $s$, one obtains a logarithmic dependence on $m_t$ multiplying $\int d^4x\,G_{\mu\nu}^a G_{\mu\nu}^a$.

\subsection*{5. Evaluating the Gamma and Color Traces}

For a more explicit look, recall
$

  (\gamma^\mu D_\mu)^2
  \;=\;
  \gamma^\mu \gamma^\nu\,D_\mu D_\nu
  \;=\;
  D^2
  ~+~
  \tfrac{1}{2}\,\sigma^{\mu\nu}\,\bigl[D_\mu,D_\nu\bigr].

$
Since $[D_\mu,D_\nu] = i\,g_s\,T^a\,G_{\mu\nu}^a$, we get
$

  (\gamma^\mu D_\mu)^2
  \;=\;
  D^2
  ~+~
  \tfrac{i\,g_s}{2}\,\sigma^{\mu\nu}\,T^a\,G_{\mu\nu}^a.

$
The Dirac traces of combinations of $\sigma^{\mu\nu}$ yield factors that pick out $G_{\mu\nu}^a\,G_{\mu\nu}^a$, while color traces over $T^a T^b$ yield factors proportional to $\delta^{ab}$. Altogether, this leads to a local expression of the form
$

  \int d^4x\,G_{\mu\nu}^a\,G_{\mu\nu}^a,

$
with a calculable coefficient (often containing $\ln m_t$ or similar).

\subsection*{6. Final Form of the Effective Action}

After dropping constant terms and higher-dimensional operators (those suppressed by additional factors of $1/m_t$), one finds the low-energy effective action:
$

  S_{\text{eff}}[A_\mu]
  \;=\;
  \int d^4x\,\Bigl[
    \tfrac{1}{4}\,G_{\mu\nu}^a\,G_{\mu\nu}^a
    \;+\;
    \delta c\;\bigl[G_{\mu\nu}^a\,G_{\mu\nu}^a\bigr]
    \;+\;\ldots
  \Bigr],

$
where $\delta c$ is a calculable coefficient depending on $m_t$, renormalization scheme, etc. Physically, this manifests as a shift in the effective gauge coupling at scales below $m_t$---the heavy quark no longer appears as a real degree of freedom but has left behind a local operator in the effective theory.

\subsection*{7. Summary}

\begin{itemize}
\item Integrating out a heavy quark in QCD amounts to computing the determinant $\det(\gamma^\mu D_\mu + m_t)$ in the background of the gluon fields.
\item At low energies ($p \ll m_t$), one expands this determinant in powers of $1/m_t$.
\item The first nontrivial, gauge-invariant local operator at dimension four is $G_{\mu\nu}^a\,G_{\mu\nu}^a$, which corrects the gluon kinetic term via a coefficient $\propto \ln(m_t)$.
\item This is the standard result in effective field theory: heavy colored fermions ``decouple'' at low energies, leaving behind local operators in the effective action.
\end{itemize}

\end{document}