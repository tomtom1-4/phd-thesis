%*******************************************************
% Table of Contents
%*******************************************************
\pagestyle{scrheadings}
%\phantomsection
\pdfbookmark[1]{\contentsname}{tableofcontents}
\setcounter{tocdepth}{2} % <-- 2 includes up to subsections in the ToC
\setcounter{secnumdepth}{3} % <-- 3 numbers up to subsubsections
\manualmark
\markboth{\spacedlowsmallcaps{\contentsname}}{\spacedlowsmallcaps{\contentsname}}
\tableofcontents
\automark[section]{chapter}
\renewcommand{\chaptermark}[1]{\markboth{\spacedlowsmallcaps{#1}}{\spacedlowsmallcaps{#1}}}
\renewcommand{\sectionmark}[1]{\markright{\textsc{\thesection}\enspace\spacedlowsmallcaps{#1}}}
%*******************************************************
% List of Figures and of the Tables
%*******************************************************
\clearpage
% \pagestyle{empty} % Uncomment this line if your lists should not have any headlines with section name and page number
\begingroup
    \let\clearpage\relax
    \let\cleardoublepage\relax
    %*******************************************************
    % List of Figures
    %*******************************************************
    %\phantomsection
    %\addcontentsline{toc}{chapter}{\listfigurename}
    %\pdfbookmark[1]{\listfigurename}{lof}
    %\listoffigures

    %\vspace{8ex}

    %*******************************************************
    % List of Tables
    %*******************************************************
    %\phantomsection
    %\addcontentsline{toc}{chapter}{\listtablename}
    %\pdfbookmark[1]{\listtablename}{lot}
    %\listoftables

    %\vspace{8ex}
    % \newpage

    %*******************************************************
    % List of Listings
    %*******************************************************
    %\phantomsection
    %\addcontentsline{toc}{chapter}{\lstlistlistingname}
    %\pdfbookmark[1]{\lstlistlistingname}{lol}
    %\lstlistoflistings

    %\vspace{8ex}

    %*******************************************************
    % Acronyms
    %*******************************************************
    \phantomsection
    \pdfbookmark[1]{Acronyms}{Acronyms}
    \markboth{\spacedlowsmallcaps{Acronyms}}{\spacedlowsmallcaps{Acronyms}}
    \chapter*{Acronyms}
    \begin{acronym}[NANOGrav]
    	\acro{SM}{Standard model}
        \acro{VEV}{Vacuum expectation value}
        \acro{SSB}{Spontaneous symmetry breaking}
        \acro{PDF}{Parton distribution function}
        \acro{QCD}{Quantum chromodynamics}
        \acro{QED}{Quantum electrodynamics}
        \acro{RGE}{Renormalization group equation}
        \acro{LO}{Leading order}
        \acro{DR}{Dimensional regularization}
        \acro{UV}{Ultraviolet}
        \acro{IR}{Infrared}
        \acro{LO}{Leading order}
        \acro{NLO}{Next-to-leading order}
        \acro{NNLO}{Next-to-next-to-leading order}
        \acro{OS}{On-shell renormalization}
        \acro{RG}{Renormalization group}
        \acro{RGE}{Renormalization group equation}
        \acro{LHC}{Large hadron collider}
        \acro{FS}{Flavor scheme}
        \acro{MC}{Monte-Carlo}
        \acro{LME}{Large mass expansion}
        \acro{HEL}{High-energy limit}
        \acro{HTL}{Heavy-top limit}
        \acro{rHTL}{Rescaled heavy-top limit}
        \acro{1PI}{One-particle irreducible}
        \acro{SCET}{Soft-collinear effective theory}
        \acro{IBP}{Integration-by-parts}
    \end{acronym}

	\vfill
	\newpage
	\pdfbookmark[1]{Notation, Constants and Conventions}{notation}
	\chapter*{Notation, Constants and Conventions} \label{chap:notation_and_conventions}
    \begin{itemize}
        \item In this thesis, I will be using the \textit{Einstein summation convention}, by which any index---be it a Lorentz, color or flavor index---which appears twice is implicitly summed over.
        \item I will be using natural units
        \begin{equation}
            \hbar = c = 1.
        \end{equation}
        \item The electron charge is
        \begin{equation}
            -e, \quad e > 0.
        \end{equation}
        \item The metric is
        \begin{equation}
            g_{\mu \nu} = \begin{pmatrix}
                1 & & & \\
                  & -1 & & \\
                  & & -1 & \\
                  & & & -1
            \end{pmatrix}.
        \end{equation}
        \item The normalization of the Levi-Civita anti-symmetric tensor $\epsilon_{\mu \nu \rho \sigma}$ is
        \begin{equation}
            \epsilon_{0123} = +1
        \end{equation}
        \item The Pauli matrices are defined as
        \begin{equation}
            \sigma^1 \equiv \begin{pmatrix}
                0 & 1 \\
                1 & 0
            \end{pmatrix}, \quad \sigma^2 = \begin{pmatrix}
                0 & - i \\
                i & 0
            \end{pmatrix}, \quad \sigma^3 = \begin{pmatrix}
                1 & 0 \\
                0 & -1
            \end{pmatrix}, \quad \tau^i \equiv \sigma^i.
        \end{equation}
        \item Unless specifically specified otherwhise we will use the following values for appearing physical constants
        \begin{itemize}
        \item $G_F =  1.16637 \times 10^{-5}\ \text{GeV}^{-2}$
        \item $m_H = 125.00 \ \GeV$
        \item $m_t = 173.06 \ \GeV$
        \item $\msMass{t} = 162.7 \ \GeV$
        \item $m_b = 4.78 \ \GeV$
        \item $\msMass{b} = 4.18 \ \GeV$
        \item $\msMass{c} =  1.27\ \GeV$
        \item $m_Z = 91.1876 \ \GeV$
        \end{itemize}
        \item We use the \texttt{NNPDF31\_nnlo\_as\_0118} and \texttt{NNPDF31\_nnlo\_as\_0118\_nf\_4}\acs{PDF} set in the 5\acs{FS} and 4\acs{FS} respectively.
        \item $\alphas$ is extracted from the \acs{PDF} set at the $Z^0$ mass.
    \end{itemize}


\endgroup
