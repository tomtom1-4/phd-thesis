%*******************************************************
% Abstract
%*******************************************************
%\renewcommand{\abstractname}{Abstract}
%\pdfbookmark[1]{Abstract}{Abstract}
% \addcontentsline{toc}{chapter}{\tocEntry{Abstract}}
\begingroup
\let\clearpage\relax
\let\cleardoublepage\relax
\let\cleardoublepage\relax
\chapter*{Abstract}

This PhD thesis presents a comprehensive analysis of the impact of finite top- and bottom-quark masses on the Higgs production cross section via gluon-gluon fusion. Our study substantially reduces the uncertainties previously associated with these mass effects on the cross section. We also explore the implications of different mass renormalization schemes, delivering results for both the \MS\ and on-shell renormalization schemes. Furthermore, we provide an in-depth comparison between the 4-flavor and 5-flavor schemes to address the treatment of finite quark masses. An additional novel aspect of our research is the examination of how finite quark masses affect the Higgs rapidity spectrum. Finally, we offer well-founded recommendations to guide future experimental and phenomenological research in this field.

\newpage
\begin{otherlanguage}{ngerman}
%\pdfbookmark[1]{Zusammenfassung}{Zusammenfassung}
\chapter*{Zusammenfassung}
Diese Dissertation präsentiert eine umfassende Analyse des Einflusses endlicher Top und Bottom Quarkmassen auf den Wirkungsquerschnitt von hadronischer Higgsproduktion im Gluon-Gluon-Fusionskanal. Unsere Studie verringert erheblich die zuvor mit diesen Masseffekten verbundenen Unsicherheiten bezüglich des Wirkungsquerschnitts. Wir untersuchen außerdem die Auswirkungen verschiedener Massenrenormalisierungsschemata und liefern Ergebnisse sowohl für das \MS- als auch das On-Shell-Renormalisierungsschema. Darüber hinaus bieten wir einen detaillierten Vergleich zwischen dem 4-Flavor- und dem 5-Flavor-Schema, um die Behandlung endlicher Quarkmassen zu adressieren. Ein neuartiger Aspekt unserer Forschung ist zudem die Untersuchung, wie sich endliche Quarkmassen auf das Higgspapiditätsspektrum auswirken. Schließlich geben wir fundierte Empfehlungen, um zukünftige experimentelle und phänomenologische Forschung in diesem Bereich zu leiten.

\end{otherlanguage}

\endgroup

\vfill
