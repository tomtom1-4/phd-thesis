% !TEX encoding = UTF-8
% !TEX TS-program = pdflatex
% !TEX root = ../thesis.tex
% !TEX spellcheck = en_US

%************************************************
\chapter{Introduction}\label{chap:one}
%************************************************

After the great successes of quantum electrodynamics (\acs{QED}), physicist in the 1950s were looking for quantum field theoretical description of other fundamental forces, like strong and weak interactions. In 1954 Chen Ning Yang and Robert Mills proposed a theory which, similar to \acs{QED}, relied on a gauge invariance, but extended the idea to non-abelian gauge groups~\cite{Yang:1954ek}. Although these theories provided a compelling foundation for encapsulating the strong and weak interactions under the banner of local gauge invariance, they also led to a conundrum: They theorized massless gauge bosons were simply not there.

One way to resolve this problem is through the phenomenon of confinement pioneered by Murray Gell-Mann~\cite{Gell-Mann:1961omu}, and realized in quantum chromodynamics (\acs{QCD}). Here, particles charged under the \acs{QCD} gauge group are not observable at large distances, since the separation of two colored particles will eventually trigger the creation of new particles, which then combine with the old ones to form color neutral states.

In 1961, Sheldon Glashow~\cite{Glashow:1961tr} and later Abdus Salam and John Ward~\cite{Salam:1964ry} proposed a models based on the $\SU{2}\times \U{1}$ gauge symmetry, which for the first time unified the electromagnetic and the weak interaction under a single theoretical framework. However, in their proposed models, the gauge symmetry was always explicitly broken by the masses the gauge bosons. Consequently, the models turned out to be unrenormalizable.

The solution materialized in the early 1960s when several physicists---François Englert, Robert Brout~\cite{Englert:1964et}, Peter Higgs~\cite{Higgs:1966ev}, Gerald Guralnik, Carl Hagen, and Tom Kibble~\cite{Guralnik:1964eu}---proposed a mechanism wherein spontaneous symmetry breaking could provide masses to otherwise massless gauge bosons while preserving the essential structure of the underlying gauge theory. This so-called \textit{Higgs mechanism} introduced a scalar field permeating spacetime that acquires a nonzero vacuum expectation value, endowing the $W$ and $Z$ bosons with mass. Shortly afterward, Stephen Weinberg applied the Higgs-mechanism to the electroweak model and hence formulated standard model (\acs{SM}) of particle physics as we know it today\footnote{Though in his original paper his model did not include quarks.}.

A key prediction was the existence of a new particle, the Higgs boson, which would serve as empirical evidence of the models' validity. Yet, for decades, the Higgs boson remained the one missing piece of this otherwise extensively tested framework. It was not until 2012, using the unprecedented energies of the Large Hadron Collider (\acs{LHC}) at CERN, that the ATLAS~\cite{ATLAS:2012yve} and CMS~\cite{CMS:2012qbp} collaborations finally observed a new particle whose properties aligned with those predicted for the Higgs boson. This discovery not only completed the particle content of the \acs{SM} but also confirmed how massive gauge bosons could arise naturally from gauge-invariant theories.

But all is not well in the \acs{SM}. Although it is one of, if not the best tested model ever proposed, the \acs{SM} has several shortcomings. For one, astrophysical and cosmological observations, have collected an overwhelming amount of evidence for the existence of a new type of very weakly interacting matter, elusively called dark matter. However, currently there is no member of the \acs{SM} which could serve as a dark matter candidate. Since the masses of all fundamental particles in the \acs{SM} are generated through the Higgs mechanism, it is plausible that at least a fraction of the \acs{DM}'s mass is generated in the same fashion. One way to look for \acs{DM}, is therefore by searching for \textit{invisible Higgs decays}~\cite{ATLAS:2017nyv, CMS:2016dhk}.

Another issue of the \acs{SM} is related to the value of the Higgs mass itself. As it stands, radiative corrections result in an unstable Higgs potential~\cite{Degrassi:2012ry}, meaning that at some point there must be physics beyond the \acs{SM}, which will at meta-stabilize the potential. This implies that our universe could be in a meta-stable state and at some point in time transition into a vastly different universe.

Since that means that the Higgs is sensitive to some new physics scale, it also gives rise to another problem---the electroweak hierarchy problem. In a nutshell, it poses the question of why the Higgs mass is so much smaller that the \text{Planck scale}. If one were to believe that the \acs{SM} is accurate up to the Planck scale, then perturbative corrections of the Higgs mass, would make the bare Higgs mass quadratically sensitive to that scale. Hence, the renormalization constant must be fine-tuned to cancel this huge number, and render a renormalized mass at the electroweak scale. This huge cancellation seems unnatural to many physicists\footnote{Myself excluded.}, and sparked lots of ideas for extensions of the \acs{SM}, such as \textit{supersymmetry}, \textit{conformal models}, or \textit{extra dimensions}. It should be noted though, that all these models have been either excluded or seen significant tension.

These selected examples demonstrate, that Higgs research has not stopped with its discovery. On the contrary, the Higgs boson has proven to be a central figure in the quest for physics beyond the Standard Model (\acs{BSM}). However, the lack of direct evidence of new physics at the \acs{LHC} so far, is likely going to remain in the near future. Physicists are therefore looking for more subtle ways to probe nature, in the hope that \acs{BSM} physics can hide behind small deviations from the \acs{SM}. Since the Higgs discovery, the field of particle physics has thus transitioned into an era of precision measurements. This requires better colliders and detectors, as well as highly sophisticated measurement strategies. But any comparison to the \acs{SM} is rendered mute, if its predictions are not precise enough. Aware of this, theoretical particle physicists have launched a large program in the pursuit of improving the accuracy of the \acs{SM} predictions.

A prime example of this pursuit can be seen in a central observable of Higgs physics phenomenology: \textit{the gluon-gluon fusion Higgs production cross section}. First predicted by Gerogi, Glashow, Mechacek and Nanopoulos in 1978~\cite{Georgi:1977gs} at leading order (\acs{LO}), it has since been calculated at next-to-\acs{LO} (\acs{NLO})~\cite{Dawson:1990zj,Djouadi:1991tka}, next-to-\acs{NLO} (\acs{NNLO})~\cite{Catani:2001ic, Harlander:2002wh,Anastasiou:2002yz} and even next-to-\acs{NNLO} (\NNNLO)~\cite{Anastasiou:2015vya,Mistlberger:2018etf}, though the latter two were computed under the assumption of an infinitely heavy top-quark. Today's cross section prediction is almost three times as large as what was originally proposed by Glashow et al., highlighting the importance of perturbative corrections in \acs{QCD}. However, the story does not end here. The cross section has been pushed to next-to-\NNNLO\ (N${}^4$LO) in the soft-virtual approximation~\cite{Das:2020adl}, such that missing higher order uncertainties are now believed to be below $1.9\%$ of the total cross section. Furthermore, electroweak corrections have been addressed both at \acs{LO}~\cite{Aglietti:2004nj,Degrassi:2004mx} and \acs{NLO} \acs{QCD}~\cite{Actis:2008ts, Actis:2008ug,Anastasiou:2008tj,Anastasiou:2018adr,Bonetti:2018ukf,Becchetti:2020wof}. Finally, the cross section has been calculated without the assumption of an infinitely heavy top-quark at \acs{NNLO}~\cite{Czakon:2021yub}.

Besides the missing higher order uncertainties, the largest source of uncertainty now stems from the light-quark induced Higgs production contribution. Here, the Higgs does not couple to the top-quark, but to lighter quarks, like the bottom- or charm-quark. Since the coupling strength to the Higgs is proportional to the mass of the respective particle, these contributions are generally suppressed with respect to the top-quark induced production channel. Nevertheless, for precision predictions, their contribution can not be omitted. Recently, we reported results on this contribution at \acs{NNLO}~\cite{Czakon:2023kqm, Czakon:2024ywb}, thereby eliminating this piece of uncertainty.

In this PhD thesis, follow up on our previous work, explaining in detail the challenges of the computation and how we overcame them. Furthermore, we combine our findings with the above-mentioned contributions, and thoroughly investigate the associated uncertainty, to give a new best prediction for the gluon-gluon fusion cross section. This shall prove useful for any further research in Higgs physics phenomenology.

This thesis is structured as follows: In chapter~\ref{chap:two}, we explain the basics of the \acs{SM} and the Higgs mechanism. We also provide a roadmap on how to perform cross section calculation. The experienced reader may skip this chapter. Chapter~\ref{chap:four} is dedicated to the Higgs production cross section. We present simple \acs{LO} and \acs{NLO} cross section calculations. These introduce core concepts which will prove useful when going to \acs{NNLO}. Since the heavy top-limit is so central for the gluon-gluon fusion cross section, we also thoroughly discuss it here. We conclude the chapter by providing an overview on the current status of the cross section. In chapter~\ref{chap:5:computational_details}, we discuss the computational details for the \acs{NNLO} computation, before providing our results in chapter~\ref{chap:six}. Finally, we give our conclusions and further outlook in chapter~\ref{chap:seven}.
