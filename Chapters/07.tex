% !TEX encoding = UTF-8
% !TEX TS-program = pdflatex
% !TEX root = ../thesis.tex


%************************************************
\chapter{Conclusions \& Outlook}\label{chap:seven}
%************************************************
\section{Conclusions}
In this work, we investigated the effect of finite bottom-quark masses on the Higgs production cross section in the gluon-gluon fusion channel. We computed the top-bottom interference contribution to the cross section at \acs{NNLO}. In our recommended setup, where the bottom-quark mass is renormalized in the \MS\ scheme and the bottom quark is included in the \acs{PDF}s, the \acs{NNLO} correction turned out to be tiny ($+0.02\ \mathrm{pb}$ at 13~TeV), however the scale uncertainties were reduced significantly. This marks the final cornerstone in a long series of theoretical improvements to the cross section. The theoretical uncertainty is now at the level $2\%$ of the total cross section.

We also investigated the effects of alternative renormalization schemes, both for the top- and for the bottom-quark mass. We found that the cross section is quite insensitive to the renormalization scheme of the top-quark mass. For the bottom-quark mass, on the other hand, the renormalization scheme proved crucial. The \acs{OS} scheme failed to exhibit good perturbative convergence at \acs{NNLO}, whereas the \MS\ scheme succeeded. Nevertheless, the cross section predictions in the two renormalization schemes turned out to be compatible within a 2-$\sigma$ interval. This analysis removed the scheme ambiguity previously associated with the cross section contribution.

Additionally, we examined the impact of the chosen \acs{FS}. This was particularly interesting for this process because it altered the treatment of the bottom-quark mass whenever it was not coupled to the Higgs. Our findings showed that the results in the 4 and 5\acs{FS} were compatible, thus providing empirical evidence that the treatment of the bottom-quark mass in the 5\acs{FS} is indeed justified.

Besides the total cross section, we also provided results for differential cross sections, namely for the Higgs-$p_T$ and -rapidity spectra. Our results for the $p_T$ spectrum aligned with what was found in previous studies~\cite{Caola:2018zye,Bonciani:2022jmb}. The top-bottom interference contribution as well as the effect of finite top-quark masses on the \acs{NNLO} Higgs rapidity distribution represent new results. The combined effect shifted the \acs{rHTL} results by around $-4\%$ across the entire rapidity spectrum.

Our analysis allowed us to give well-grounded recommendations for future research in this field. Finally, we combined our findings with other state-of-the-art computations and carefully assessed their theoretical uncertainties. The resulting prediction can be used for future experimental or phenomenological applications.

\section{Outlook}
Since the theoretical uncertainty is now dominated by scale uncertainties, and this despite the fact that the cross section has been computed at partial N${}^4$LO, it is unlikely that we will see a further significant reduction of the total theory uncertainty. The only uncertainties that will probably be reduced in the near future are the \acs{MHOU} in the \acs{PDF}s, as well as the uncertainty on the soft-virtual approximation. The former partly arises from the incomplete \NNNLO\ DGLAP evolution kernels, \ie\ once these have been worked out, we can see a further reduction of the error. However, it is unclear how much the uncertainty will actually improve, as the \acs{MHOU} also include the uncertainties of the perturbative cross sections the \acs{PDF}s are matched to. Moreover, even if the uncertainty were completely removed, the impact on the total error of the cross section is almost negligible. Regarding the error of the soft-virtual approximation, the latest \textit{Les Houch wishlist}~\cite{Huss:2025nlt}, named the N${}^4$LO Higgs production cross section as a major aim, and with tremendous progress in the calculation of the necessary amplitudes recently~\cite{Chen:2025utl, mistlberger2025rvvtimesvinterferencecontributionsinclusive}, we can hope that the full N${}^4$LO cross section will become available in the coming years. And although the total uncertainty is likely not going to be reduced significantly, the full calculation is still highly desirable, in order to assess the quality of the soft-virtual approximation at this order.

The most promising way to make significant progress in the mitigation of the theory uncertainties, is by improving on the \acs{PDF} and $\alphas$ uncertainties. We have seen in Section~\ref{subsec:4:pdf_uncertainties} that the \acs{PDF}-uncertainty estimates of the \texttt{NNPDF} and \texttt{MSHT} collaborations are vastly different. If the small uncertainties of the \texttt{NNPDF} collaboration are to be believed, then the \acs{PDF}$ + \alphas$ uncertainty, which is dominating for this progress, would be reduced by another factor of 2! Understanding the origin of this discrepancy could be key, not only for the Higgs production cross section, but for precision predictions in general.

The low-$p_T$ region of the top-bottom interference contribution is particularly interesting since here the effects from finite bottom-quark masses are most noticeable. Consequently, it is the region most suited for the determination of the bottom-Yukawa coupling. The fixed-order $p_T$-distribution is however only reliable above $p_T \sim 20 \ \mathrm{GeV}$. Below, large logarithms can no longer be ignored and require the resummation to all orders. The origin and resummation of these logarithms is not yet well understood and more research along the lines of Refs.~\cite{Caola:2018zye, Liu:2017vkm} is needed to improve our understanding as well as the precision of the cross section in this critical region of the phase space.

In experimental searches and measurements of Higgs boson properties, one typically considers only specific production and decay channels of the Higgs. Since the decay width of the Higgs is orders of magnitude smaller than its mass $\Gamma_H /m_H = 3 \times 10^{-5}$, one can apply the narrow width approximation (see Eq.~\eqref{eq:4:narrow_width_approximation}), such that production and decay of the Higgs boson become factorized. Our results therefore serve as a stepping stone for a more precise determination of the $pp \rightarrow H \rightarrow X$ cross sections. This will be crucial, especially in the light of the high luminosity phase of the \acs{LHC}.