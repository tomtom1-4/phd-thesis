% !TEX encoding = UTF-8
% !TEX TS-program = pdflatex
% !TEX root = ../thesis.tex

%************************************************
\chapter{The Standard Model of Particle Physics}\label{chap:two}
%************************************************

\section{Electroweak Symmetry breaking}
The standard model (\acs{SM}) of particle physics is a theory describing all known matter and their fundamental interactions except for gravity. It unifies the electromagnetic, weak, and strong forces under a single theoretical framework.
\begin{figure}
\centering
\includegraphics[width=0.7\textwidth]{Images/SM.pdf}
\caption{Elementary particles of the \acs{SM}. The was image generated with the help of Ref.~\cite{Neutelings_2024}.}
\label{fig:2:SM}
\end{figure}
The matter content of the \acs{SM} is classified into two primary groups: \textit{fermions} and \textit{bosons}. The fermions have spin $1/2$, they are further subcategorized into \textit{quarks} and \textit{leptons}. Quarks participate in strong interactions, while leptons interact only via the electromagnetic and weak forces. In contrast, bosons have integer spin. There exists a single particle with spin 0, the \textit{Higgs} boson, and four vector bosons, namely the gluon, the photon, and the $W$ and $Z$ boson. The vector bosons act as force carrier for the strong, the electromagnetic and the weak force respectively. Fermions are organized into three generations, with each generation containing two types of quarks (up-type and down-type) and two leptons (a charged lepton and its corresponding neutrino). These generations are shown in Fig.~\ref{fig:2:SM}, along with their masses, charges, and spins.

The interactions between \acs{SM} particles are described by a non-abelian gauge theory of the $\SU{3}_C \times \SU{2}_L \times \U{1}_Y$ group. Here $\SU{3}_C$ governs the strong interactions. It applies to all \textit{colored} particles, \ie~quarks and gluons. The quarks transform under the fundamental representation of the $\SU{3}_C$ group. $\SU{2}_L \times \U{1}_Y$ governs electroweak interactions. The $\SU{2}_L$ transformation acts non-trivially only on \textit{left-handed} fermions which form doublets
\begin{equation}
\begin{gathered}
L_{iL} \equiv \begin{pmatrix}
\nu_{iL} \\
l_{iL}
\end{pmatrix}, \quad Q_{iL} \equiv \begin{pmatrix}
u_{iL} \\
d_{iL}
\end{pmatrix}, \\
\nu_i = \left( \nu_e, \nu_\mu, \nu_\tau \right),\quad  l_i = \left( e, \mu, \tau \right),\quad u_i = \left(u, c, t \right),\quad d_i = \left(d, s, b \right).
\end{gathered}
\end{equation}
The phase transformation $\U{1}_Y$ acts on all particles except neutrinos according to their quantum number, the \textit{hypercharge} $Y$. The symmetry is spontaneously broken to $\SU{3}_C \times \U{1}_Q$, where the $\U{1}_Q$ group corresponds to gauge transformation of the electromagnetic interaction, hence the subscript $Q$ for the \textit{electric charge}. To ensure that the particles have the correct charges, the hypercharge must satisfy the \textit{Gell-Mann--Nishijima relation}:
\begin{equation}
\frac{Y}{2} = Q - I^3.
\end{equation}
With the particle charges displayed in Fig.~\ref{fig:2:SM} we then get
\begin{equation*}
\begin{tabular}{ccccccc}
  \hline
                & $L_{iL}$ & $Q_{iL}$ & $\nu_{iR}$ & $l_{iR}$ & $u_{iR}$ & $d_{iR}$ \\ \hline
  $\frac{Y}{2}$ & $-\frac{1}{2}$ & $\frac{1}{6}$ & $0$ & $-1$ & $\frac{2}{3}$ & $-\frac{1}{3}$ \\ \hline
\end{tabular}
\end{equation*}

The transformation properties of the gauge bosons is dictated by the covariance of the covariant derivative
\begin{equation}
\begin{gathered}
D_\mu \equiv \partial_\mu - i g A^a_\mu T_R^a - i g_2 W_\mu^a I^a + i g_Y \frac{Y}{2} B_\mu \\
T_R^a = \begin{cases} T^a \quad &\text{for quarks}, \\
  0 \quad &\text{for leptons},
\end{cases}
\qquad I^a = \begin{cases} \frac{\tau}{2} \quad &\text{for left-handed fermions}, \\
  0 \quad &\text{for right-handed fermions},
\end{cases}
\end{gathered}
\end{equation}
where $T^a$ and $\tau^a$ are the \textit{Gell-Mann} and \textit{Pauli} matrices.

Before spontaneous symmetry breaking, the Lagrangian which governs the evolution of all matter fields must be invariant under the $\SU{3}_C \times \SU{2}_L \times \U{1}_Y$ gauge group. Up to a \Charge\Parity\ violating term\footnote{The absence of the \Charge\Parity\ violating term $\theta \frac{g^2}{64 \pi^2}\epsilon^{\mu \nu \alpha \beta} F^{a}_{\mu \nu} F^{a}_{\alpha \beta}$ is an unsolved problem of particle physics, known as the strong CP problem.} the SM Lagrangian is the most general mass-dimension four Lagrangian for the described particle content
\begin{equation}
\mathcal{L}_\mathrm{SM} = \mathcal{L}_G + \mathcal{L}_F + \mathcal{L}_Y + \mathcal{L}_H.
\end{equation}
The gauge-field Lagrangian $\mathcal{L}_G$ describes the free propagation and in the case of the non-abelian groups $\SU{3}_C$ and $\SU{2}_L$ also the self-interaction of the gauge bosons. It is given by
\begin{equation}
\begin{gathered}
\mathcal{L}_G = - \frac{1}{4} G^{a}_{\mu \nu} G^{a\, \mu\nu} - \frac{1}{4} W^a_{\mu \nu} W^{a\, \mu\nu} - \frac{1}{4} B_{\mu\nu} B^{\mu \nu}, \\
G^a_{\mu\nu} \equiv \partial_\mu A^a_\nu - \partial_\nu A^a_\mu + g f^{abc} A^b_\mu A^c_\nu, \\
W^a_{\mu\nu} \equiv \partial_\mu W^a_\nu - \partial_\nu W^a_\mu + g_2 \epsilon^{abc} W^b_\mu W^c_\nu, \\
B_{\mu\nu} \equiv \partial_\mu B_\nu - \partial_\nu B_\mu.
\end{gathered}
\end{equation}
The propagation of the fermions and their interaction with the gauge bosons is described by
\begin{equation}
\mathcal{L}_F = \bar{L}_{iL} i \slashed{D} L_{iL} + \bar{\nu}_{iR} i \slashed{D} \nu_{iR} + \bar{l}_{iR} i \slashed{D} l_{iR} + \bar{Q}_{iL} i \slashed{D} Q_{iL} + \bar{u}_{iR} i \slashed{D} u_{iR} + \bar{d}_{iR} i \slashed{D} d_{iR}.
\end{equation}

The Higgs field is a doublet of the $\SU{2}_L$ group. We want the field to have a non-vanishing vacuum expectation value (\acs{VEV}) to dynamically generate the fermion and boson masses. Of course, the vacuum cannot carry an electric charge, which means that the Higgs field must be electrically neutral along the direction of \textit{spontaneous symmetry breaking} (\acs{SSB}). We choose this to be the second component of the doublet. With the Gell-Mann--Nishijima relation we can then deduce that hypercharge of the doublet must be $Y = +1$. The Higgs doublet field thus takes the form
\begin{equation}
\Phi = \begin{pmatrix}
  \phi^+ \\
  \phi^0
\end{pmatrix},
\end{equation}
where the superscript indicates the electic charge.

In order to generate a non-vanishing \acs{VEV}, the Higgs field must be in a potential with a global minimum away from zero. Hence, the only gauge invariant mass-dimension four Lagrangian we can construct is
\begin{equation}
\begin{gathered}
\mathcal{L}_H = \left( D_\mu \Phi \right)^\dagger \left( D^\mu \Phi \right) - V(\Phi) \\
V(\Phi) = \lambda (\Phi^\dagger \Phi )^2 - \mu^2 \Phi^\dagger \Phi, \quad \mu^2, \lambda > 0.
\end{gathered}
\end{equation}
The minimum of the Higgs potential $V$ is at
\begin{equation}
\Phi_0^\dagger \Phi_0 = \frac{\mu^2}{2 \lambda} \equiv \frac{v^2}{2} \neq 0.
\end{equation}
After (\acs{SSB}) we can expand the Higgs field around its minimum
\begin{equation}
\Phi = \begin{pmatrix}
  \phi^+ \\
  \frac{1}{\sqrt{2}} ( v + H + i \xi ).
\end{pmatrix}
\end{equation}
The real scalar field $H$ is the famous Higgs boson, whereas the fields $\phi^\pm$ and $\xi$ are unphysical since they can always be eliminated through a gauge transformation (\textit{would-be Goldstone bosons}). After inserting the expansion in the Higgs Lagrangian, the mass of the Higgs can be read off from its square term
\begin{equation}
m_H = \sqrt{2} \mu.
\end{equation}

\acs{SSB} enables the generation of vector boson masses without breaking the gauge symmetry explicitly. If we insert the expanded Higgs field in the Higgs Lagrangian, we get quadratic terms of the gauge boson fields
\begin{equation}
\begin{split}
\mathcal{L}_H &\supsetneq \frac{v^2}{2} \bigg \lbrace g_2^2 \begin{pmatrix} 0 & 1 \end{pmatrix} I^a I^b \begin{pmatrix} 0 \\ 1 \end{pmatrix} W^a_\mu W^{b\, \mu} - g_2 g_Y \begin{pmatrix} 0 & 1 \end{pmatrix} I^a \begin{pmatrix} 0 \\ 1 \end{pmatrix} W_\mu^a B^\mu + \frac{g_Y^2}{4} B_\mu B^\mu \bigg \rbrace \\
&= \frac{v^2}{2} \bigg \lbrace \frac{g_2^2}{4} \left[ (W^1)^2 + (W^2)^2 \right] + \frac{1}{4} \begin{pmatrix} B^\mu & W^{3\, \mu} \end{pmatrix} \begin{pmatrix}  g_Y^2 & g_Y g_2 \\ g_Y g_2 & g_2^2 \end{pmatrix} \begin{pmatrix} B_\mu \\ W^3_\mu \end{pmatrix} \bigg \rbrace.
\end{split}
\end{equation}
By diagonalizing the mass matrix we obtain the physical states
\begin{equation}
\begin{gathered}
\begin{pmatrix}
A^\gamma_\mu \\
Z_\mu
\end{pmatrix} = \begin{pmatrix}
\cos \theta_W & - \sin \theta_W \\
\sin \theta_W & \cos \theta_W
\end{pmatrix} \begin{pmatrix}
B_\mu \\
W_\mu^3
\end{pmatrix}
\end{gathered}, \quad \cos \theta_W = \frac{g_2}{\sqrt{g_Y^2 + g_2^2}}, \sin \theta_W = \frac{g_Y}{\sqrt{g_Y^2 + g_2^2}}.
\end{equation}
In this new basis, we have one massless boson $A^\gamma_\mu$, which we identify as the photon and a charge neutral boson of mass
\begin{equation}
m_Z = \frac{v}{2} \sqrt{g_Y^2 + g_2^2}.
\end{equation}
The vector bosons $W^1$ and $W^2$ are not eigenstates of the charge operator. We therefore define the new states
\begin{equation}
W^\pm_\mu = \frac{1}{\sqrt{2}} \left( W^1_\mu \mp i W^2_\mu \right), \qquad Q W_\mu^\pm = \pm W_\mu^\pm,
\end{equation}
which are eigenstates of $Q$ and have mass
\begin{equation}
m_W = \frac{v}{2} g_2.
\end{equation}

Last but not least, we discuss the Yukawa sector of the \acs{SM} Lagrangian. Before \acs{SSB}, fermions cannot generate masses because a mass term would mix the left- and right-handed components of the fields, thereby breaking the chiral gauge symmetry. Here, once again, the Higgs field comes to the rescue: by coupling the fermions with the Higgs field through a Yukawa interaction\footnote{In the original formulation of the \acs{SM}, there are no neutrino Yukawa interactions, since they were believed to be massless. Neutrino oscillation experiments have shown however, that neutrinos do in fact have finite masses.}
\begin{equation}
\mathcal{L}_Y = - \left( y_{ij}^\nu \bar{L}_{iL} \Phi^c \nu_{jR} + y_{ij}^l \bar{L}_{iL} \Phi l_{jR} + y_{ij}^d \bar{Q}_{iL} \Phi d_{jR} + y_{ij}^u \bar{Q}_{iL} \Phi^c u_{jR} \right) + \hc,
\end{equation}
where $\Phi^c$ is the charge-conjugated field to $\Phi$, we do not explicitly break the symmetry. However, after \acs{SSB} this Lagrangian will generate exactly the required mixing between left- and right-handed fields to generate the fermion masses. The Yukawa-interaction matrices $y_{ij}^{\nu,l, d, u}$ can be shifted from the Yukawa sector to the fermion sector through a field redefinition. Indeed, if we apply the \textit{singular value decomposition} of the Yukawa matrix
\begin{equation}
y = U_L^\dagger y_{\mathrm{diag}} U_R, \quad \text{with} \quad (y_{\mathrm{diag}})_{ij} = m_i \delta_{ij} \quad \text{and} \quad U_{L,R} \in \U{3},
\end{equation}
and redefine our fermion fields to be
\begin{equation}
f_{iR} \longrightarrow U_{Rij} f_{jR}, \qquad f_{iL} \longrightarrow U_{Lij} f_{jL}, \qquad f = \nu, l, u, d
\end{equation}
the Yukawa Lagrangian becomes
\begin{equation}
\mathcal{L}_Y = - \sum_{i}\left( m_{\nu_i} \bar{\nu}_i \nu_i + m_{l_i} \bar{l}_i l_i + m_{u_i} \bar{u}_i u_i + m_{d_i} \bar{d}_i d_i \right)  \left(1 + \frac{H}{v} \right).
\end{equation}
As an immediate consequence, we observe that the Yukawa coupling of the Higgs to the fermions is proportional to the mass of that fermion.
The field redefinition is a change from a flavor eigenbasis, which is diagonal in the couplings to the gauge bosons, to a mass eigenbasis. In the mass eigenbasis the part of fermion Lagrangian which contains the interaction to the electroweak gauge bosons after SSB is
\begin{equation}
\begin{split}
\mathcal{L}_F \supsetneq &\sum_f (-Q_f) e \bar{f}_i \slashed{A}^\gamma f_i + \sum_f \frac{e}{\sin \theta_W \cos \theta_W} \bar{f}_i ( I_f^3 \gamma^\mu P_L - \sin^2 \theta_W Q_f \gamma^\mu) f_i Z_\mu \\
&+ \frac{e}{\sqrt{2} \sin \theta_W} \left( \bar{u}_i \gamma^\mu P_L (V_\mathrm{CKM})_{ij} d_j W_\mu^+ + \bar{d}_i \gamma^\mu P_L (V_{\mathrm{CKM}}^\dagger)_{ij} u_j W_\mu^- \right) \\
&+ \frac{e}{\sqrt{2} \sin \theta_W} \left( \bar{\nu}_i \gamma^\mu P_L (V_\mathrm{PMNS}^\dagger)_{ij} l_j W_{\mu}^+ + \bar{l}_i \gamma^\mu P_L (V_\mathrm{PMNS})_{ij} v_j W_\mu^- \right).
\end{split}
\end{equation}
Here we identified the electromagnetic coupling constant
\begin{equation}
e = \frac{g_2 g_Y}{\sqrt{g_2^2 + g_Y^2}},
\end{equation}
as the factor in front of the photon interaction term. The operators $P_{L,R}$ are just the projectors onto the left- and right-handed components
\begin{equation}
P_{L,R} = \frac{1 \mp \gamma^5}{2}.
\end{equation}
The \textit{CKM} and \textit{PMNS matrices}\footnote{Named after Cabibbo, Kobayashi and Maskawa, and Pontecorvo, Maki, Nakagawa and Sakata.} are the results of the field redefinitions
\begin{equation}
V_\mathrm{CKM} \equiv {U_L^{u}}^\dagger U_L^d, \qquad V_\mathrm{PMNS} \equiv {U_L^{l}}^\dagger U_L^\nu.
\end{equation}
Typically, one prefers to work in the mass eigenbasis of the quarks, while the neutrinos are kept in the flavor eigenbasis, in which case one encounters flavor changes (\textit{neutrino oscillations}) trough propagation. This is why the PMNS matrix is defined in terms of the complex conjugate of the CKM matrix equivalent in the lepton sector.



\section{Cross Sections}
Many of the great successes of the \acs{SM} are its \textit{cross section} predictions. The cross section is simply defined as the probability to create some final state from some initial state per unit of time per target particle divided by the incoming particle flux. This means that the cross section can be easily measured with a simple counting experiment. Experiments like \texttt{CMS} of \texttt{Atlas} do exactly that: they smash particles together and count how many times a certain final state was produced in some time interval. On the theory side, the cross section can be calculated with
\begin{equation}
\dd \hat{\sigma}_{ij \rightarrow n} = \frac{1}{F} \dd \Phi_n |M_{ij\rightarrow n}|^2,
\end{equation}
where $F$ is the \textit{flux factor}\footnote{In the following we assume that the initial state particles are massless.}
\begin{equation}
F \equiv 4 p_1 \cdot p_2,
\end{equation}
$\dd \Phi_n$ is the \textit{Lorentz invariant phase space measure}
\begin{equation}
\dd \Phi_n = \prod_{i = 1}^n \frac{\dd^3 \mathbf{q}_i}{(2 \pi)^3 2 q_i^0} (2 \pi)^4 \delta^{(4)} \! \left(p_1 + p_2 - \sum_{i = 1}^n q_i \right),
\end{equation}
and $M_{fi}$ is the \textit{scattering amplitude} describing the hard interactions.

The computation of cross sections involves three basic steps:
\begin{enumerate}
  \item the calculation of the hard scattering amplitude,
  \item the phase-space integration,
  \item and the convolution with \textit{parton distribution functions} (\acs{PDF}s).
\end{enumerate}
In the following we will discuss them one-by-one.

\subsection{The Hard Scattering Amplitude}
The Hard Scattering Amplitude describes the transition probability from a certain initial state to a specific finial state. Since the scattering is \textbf{hard}, the energy transfer between the particles during the scattering process is large compared to the QCD scale. This means we are in the perturbative regime of QCD, and we can perform an expansion in the coupling constant
\begin{equation}
M_{ij \rightarrow n} = \alphas^{n_\text{Born}} \left( M_{ij \rightarrow n}^{(0)} + \frac{\alphas}{\pi} M_{ij \rightarrow n}^{(1)} + \left(\frac{\alphas}{\pi}\right)^2 M_{ij \rightarrow n}^{(2)} + \BigO{\alphas^3} \right).
\end{equation}
$n_\text{Born}$ is the power of the coupling constant at \textit{leading order} (\acs{LO}). The coefficients of the series are calculable graphically using \textit{Feynman rules}.
\subsection{The Phase-Space Integration}
\subsection{The Parton Distribution Functions}
In hadron collisions, the initial state is not made up of elementary particles, but are bound states thereof. This means that during an inelastic scattering event, the partons which take part in the short-range interaction only carry a fraction of the original hadron momentum
\begin{equation}
p_1 = x_1 P_1, \qquad p_2 = x_2 P_2.
\end{equation}
Here $p_1$ and $p_2$ are the momenta of the partons and $P_1$ and $P_2$ are the momenta of the hadrons. Since the momentum of the parton can not be larger than that of the hadron, $x_{1,2}$ is restricted to be less than one. Furthermore, since the energy of the parton must be positive the momentum fraction must also be positive. Otherwise, the momentum fraction is a priori unconstrained, we therefore integrate over all allowed values of $x_1$ and $x_2$
\begin{equation}
\dd \sigma_{H_1 H_2 \rightarrow n} = \sum_{i,j} \int_0^1 \dd x_1 \dd x_2 \, f_{H_1,i}(x_1, \mu_F) f_{H_2, j}(x_2, \mu_F) \dd \hat{\sigma}_{ij \rightarrow n}(x_1 P_1, x_2 P_2, \mu_F, \mu_R).
\label{eq:2:factorization}
\end{equation}
$f_{H_{k}, i}(x_k, \mu_F)$ are the \acs{PDF}s. They descibe the probability of finding a parton $i$ with momentum fraction $x_k$ inside the hadron $H_k$. Momentum conservation then requires the normalization
\begin{equation}
\sum_i \int_0^1 \dd x \, x f_{H, i} (x, \mu_F) = 1.
\end{equation}
\begin{figure}
\centering
\includegraphics[width=\figurewidth]{Images/PDF.pdf}
\caption{The various \acs{PDF}s mulitplied by $x$ as a function of $x$. The plot was created using the \texttt{LHAPDF6}~\cite{Buckley:2014ana} interface to the \texttt{NNPDF31\_nnlo\_as\_0118}~\cite{NNPDF:2017mvq} \acs{PDF} set at a scale of $\mu_F=m_H/2$.}
\label{fig:2:PDF}
\end{figure}
The factorization theorem~\eqref{eq:2:factorization} is central in the \acs{SM} as it tells us that the \acs{PDF}s are universal quantities, \ie\ they are not specific to any one process. It is a postulate of the parton model, in which hadrons are thought of as collection of the free elementary particles. In \acs{QCD} however, the theorem requires proof~\cite{Collins:1989gx}!
The \acs{PDF} for all light partons are displayed in Fig.~\ref{fig:2:PDF}. \acs{PDF}s describe long range interactions, a regime in which \acs{QCD} is non-perturbative. As such, \acs{PDF}s are non-perturbative objects which have to be measured in experiments or be calculated non-pertubatively, \eg\ on the lattice.

The scale $\mu_F$ marks the boundary, over which we treat interactions perturbatively. This scale is unphysical in the sense that it is not a parameter in our theory, nor can it be measured in an experiment. This independents can be precisely formulated in terms of the \textit{renormalization group equation} (\acs{RGE})
\begin{equation}
\frac{\dd}{\dd \ln \mu_F} f_{H,i}^B = 0,
\end{equation}
which states that the \textit{bare, unrenormalized} \acs{PDF} can not depend on the unphysical scale $\mu_F$. If we now express the unrenormalized \acs{PDF} in terms of its renormalized counterpart
\begin{equation}
f_{H,i}^B \equiv Z_{ij} \otimes f_{H,j}^R \equiv \int_0^1 \dd y \dd z\, Z_{ij}(y) f_{H,j}^R(z) \delta (x - y z),
\end{equation}
then we can use the \acs{RGE} to get the factorization scale dependents of the renromalized \acs{PDF}
\begin{equation}
\begin{gathered}
0 = \frac{\dd}{\dd \ln \mu_F} f_{H,i}^B = \frac{\dd \alphas}{\dd \ln \mu_F} \frac{\dd Z_{ij}}{\dd \alphas} \otimes f_{H,j}^R + Z_{ij} \otimes \frac{\dd f_{H,j}}{\dd \ln\mu_F} \\
\Rightarrow \frac{\dd f_{H,i}^R}{\dd \ln \mu_F} = - Z_{ij}^{-1} \otimes \left( 4 \pi \beta - 2 \epsilon \alphas \right) \frac{\dd Z_{jk}}{\dd \alphas} \otimes f_{H,k}^R =  2 \alphas Z_{ij}^{-1} \otimes \frac{\dd Z_{ij}^{(1)}}{\dd \alphas} \otimes f_{H,k}^R.
\end{gathered}
\end{equation}
Here we used the definition of the $d$-dimensional $\beta$\textit{-function}
\begin{equation}
\overline{\beta} \equiv \frac{1}{4 \pi}  \frac{\dd \alphas}{\dd \ln \mu_F} = \beta - \epsilon \frac{\alphas}{2 \pi}, \quad \beta = \frac{\alphas}{2 \pi} \frac{\dd Z^{(1)}}{\dd \ln \alphas},
\end{equation}
where $Z^{(1)}_{\alphas}$ and $Z^{(1)}_{ij}$ are the residues of the renormalization constants of the coupling constant and the \acs{PDF}s respectively. At one loop, the \acs{PDF} renormalization constant is designed to absorb the singularities from tree-level initial-state-collinear radiation, it therefore reads
\begin{equation}
Z_{ij}(z) = \delta(1 - z) \delta_{ij} + \frac{\alphas}{2 \pi} \frac{1}{\epsilon} P^{(0)}_{ij}(z),
\end{equation}
with the \textit{Altarelli-Parisi splitting kernels}
\begin{equation}
\begin{split}
&P_{qq}^{(0)} (z) = P_{\bar{q}\bar{q}}^{(0)}(z) = C_F  \left[ \frac{1 + z^2}{(1 - z)_+} + \frac{3}{2} \delta (1 - z) \right], \\[5pt]
&P_{qg}^{(0)} (z) = T_F \left[ z^2 + (1 - z)^2 \right], \\[5pt]
&P_{gq}^{(0)} (z) = C_F \left[ \frac{1 + (1 - z)^2}{z} \right], \\[3pt]
&P_{gg}^{(0)} (z) = 2 C_A \left[ \frac{z}{(1 - z)_+} + \frac{1 - z}{z} + z (1 - z) \right] + \delta (1 - z) \frac{\beta_0}{2},
\end{split}
\end{equation}
and
\begin{equation}
\beta_0 = \frac{11}{3} C_A - \frac{4}{3}T_F n_l.
\end{equation}
So even though the \acs{PDF}s are non-perturbative, their dependence on the factorization scale is perturbative and calculable with the \textit{Dokshitzter-Gribow-Lipatow-Altarelli-Parisi-evolution equation}~\cite{Dokshitzer:1977sg,Gribov:1972ri,Altarelli:1977zs}
\begin{equation}
\frac{\dd f^R_{H, i}}{\dd \ln \mu_F} = \frac{\alphas}{\pi} P_{ij}^{(0)} \otimes f^R_{H,j}.
\end{equation}





