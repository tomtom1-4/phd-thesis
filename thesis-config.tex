% ****************************************************************************************************
% 0. Set the encoding of your files. UTF-8 is the only sensible encoding nowadays. If you can't read
% äöüßáéçèê∂åëæƒÏ€ then change the encoding setting in your editor, not the line below. If your editor
% does not support utf8 use another editor!
% ****************************************************************************************************
\PassOptionsToPackage{utf8}{inputenc}
  \usepackage{inputenc}

\PassOptionsToPackage{T1}{fontenc} % T2A for cyrillics
  \usepackage{fontenc}


% ****************************************************************************************************
% 1. Configure classicthesis for your needs here, e.g., remove "drafting" below
% in order to deactivate the time-stamp on the pages
% (see ClassicThesis.pdf for more information):
% ****************************************************************************************************
\PassOptionsToPackage{
  drafting=false,    % print version information on the bottom of the pages
  tocaligned=false, % the left column of the toc will be aligned (no indentation)
  dottedtoc=true,  % page numbers in ToC flushed right
  eulerchapternumbers=false, % use AMS Euler for chapter font (otherwise Palatino)
  linedheaders=false,       % chaper headers will have line above and beneath
  floatperchapter=true,     % numbering per chapter for all floats (i.e., Figure 1.1)
  eulermath=false,  % use awesome Euler fonts for mathematical formulae (only with pdfLaTeX)
  beramono=true,    % toggle a nice monospaced font (w/ bold)
  palatino=false,    % deactivate standard font for loading another one, see the last section at the end of this file for suggestions
  style=arsclassica % classicthesis, arsclassica
}{thesis}


% ****************************************************************************************************
% 2. Personal data and user ad-hoc commands (insert your own data here)
% ****************************************************************************************************
\newcommand{\myTitle}{Finite-Quark-Mass Effects on the Higgs Production Cross Section in the Gluon-Gluon Fusion Channel\xspace}
\newcommand{\mySubtitle}{\xspace}
\newcommand{\myDegree}{\xspace}
\newcommand{\myName}{Tom Claus Rudolf Schellenberger\xspace}
\newcommand{\myProf}{Prof. Dr. Micha\l{} Czakon\xspace}
\newcommand{\myOtherProf}{Prof. Dr. Robert Harlander\xspace}
\newcommand{\mySupervisor}{Prof. Dr. Micha\l{} Czakon\xspace}
\newcommand{\myOtherSupervisor}{Prof. Dr. Robert Harlander\xspace}
\newcommand{\myFaculty}{Fakult{\"a}t f{\"u}r Mathematik, Informatik und Naturwissenschaften\xspace}
\newcommand{\myInstitute}{Institute for Theoretical Particle Physics and Cosmology\xspace}
\newcommand{\myDepartment}{Theory group\xspace}
\newcommand{\myUni}{Rheinisch-Westf{\"a}lische Technische Hochschule\xspace}
\newcommand{\myUniGenitiv}{Rheinisch-Westf{\"a}lischen Technischen Hochschule\xspace}
\newcommand{\myLocation}{Aachen\xspace}
\newcommand{\myTime}{1. Juli 2025\xspace}
\newcommand{\myVersion}{Version 1.0}

% ********************************************************************
% Setup, finetuning, and useful commands
% ********************************************************************
\providecommand{\mLyX}{L\kern-.1667em\lower.25em\hbox{Y}\kern-.125emX\@}
\newcommand{\ie}{i.e.}
\newcommand{\Ie}{I.e.}
\newcommand{\eg}{e.g.}
\newcommand{\Eg}{E.g.}
% ****************************************************************************************************


% ****************************************************************************************************
% 3. Loading some handy packages
% ****************************************************************************************************
% ********************************************************************
% Packages with options that might require adjustments
% ********************************************************************
\PassOptionsToPackage{ngerman,american}{babel} % change this to your language(s), main language last
% Spanish languages need extra options in order to work with this template
%\PassOptionsToPackage{spanish,es-lcroman}{babel}
    \usepackage{babel}

\usepackage{csquotes}

%\PassOptionsToPackage{%
  %backend=biber,bibencoding=utf8, %instead of bibtex
  %backend=bibtex, bibencoding=utf8,%
  %language=auto,%
  %style=numeric-comp,%
  %style=authoryear-comp, % Author 1999, 2010
  %bibstyle=authoryear,dashed=false, % dashed: substitute rep. author with ---
  %sorting=nyt, % name, year, title
  %sorting=none,
  %maxbibnames=10, % default: 3, et al.
  %backref=true,%
  %natbib=true % natbib compatibility mode (\citep and \citet still work)
  %style=my_style2.bst,
%}{biblatex}
%\usepackage{biblatex}%\addbibresource
%\usepackage{bibentry}

%\PassOptionsToPackage{fleqn}{amsmath}  % This moves equations away from the center      % math environments and more by the AMS
  \usepackage{amsmath}

% ********************************************************************
% General useful packages
% ********************************************************************
\usepackage{graphicx} %
\usepackage{scrhack} % fix warnings when using KOMA with listings package
\usepackage{xspace} % to get the spacing after macros right
\PassOptionsToPackage{printonlyused}{acronym} %smaller % can be added as option
  \usepackage{acronym} % nice macros for handling all acronyms in the thesis
  %\renewcommand{\bflabel}[1]{{#1}\hfill} % fix the list of acronyms --> no longer working
  %\renewcommand*{\acsfont}[1]{\textsc{#1}}
  %\renewcommand*{\aclabelfont}[1]{\acsfont{#1}}
  %\def\bflabel#1{{#1\hfill}}
  \def\bflabel#1{{\acsfont{#1}\hfill}}
  \def\aclabelfont#1{\acsfont{#1}}
\usepackage{lipsum}
% ****************************************************************************************************
%\usepackage{pgfplots} % External /PGF support (thanks to Andreas Nautsch)
%\usetikzlibrary{external}
%\tikzexternalize[mode=list and make, prefix=ext-tikz/]
% ****************************************************************************************************

\usepackage{pifont}% http://ctan.org/pkg/pifont
\newcommand{\cmark}{\ding{51}}%
\newcommand{\xmark}{\ding{55}}%


% ****************************************************************************************************
% 4. Setup floats: tables, (sub)figures, and captions
% ****************************************************************************************************
\usepackage{tabularx} % better tables
  \setlength{\extrarowheight}{3pt} % increase table row height
\newcommand{\tableheadline}[1]{\multicolumn{1}{l}{\spacedlowsmallcaps{#1}}}
\newcommand{\myfloatalign}{\centering} % to be used with each float for alignment
\usepackage{subfig}
% ****************************************************************************************************


% ****************************************************************************************************
% 5. Setup code listings
% ****************************************************************************************************
\usepackage{listings}
%\lstset{emph={trueIndex,root},emphstyle=\color{BlueViolet}}%\underbar} % for special keywords
\lstset{language=[LaTeX]Tex,%C++,
  morekeywords={PassOptionsToPackage,selectlanguage},
  keywordstyle=\color{RWTHblau},%\bfseries,
  basicstyle=\small\ttfamily,
  %identifierstyle=\color{NavyBlue},
  commentstyle=\color{Green}\ttfamily,
  stringstyle=\rmfamily,
  numbers=none,%left,%
  numberstyle=\scriptsize,%\tiny
  stepnumber=5,
  numbersep=8pt,
  showstringspaces=false,
  breaklines=true,
  %frameround=ftff,
  %frame=single,
  belowcaptionskip=.75\baselineskip
  %frame=L
}
% ****************************************************************************************************




% ****************************************************************************************************
% 6. Last calls before the bar closes
% ****************************************************************************************************

\usepackage{amsmath,amssymb}
\usepackage{pifont}
\usepackage{mathtools}
\usepackage{epsfig}
\usepackage{afterpage}
\usepackage{url}
\usepackage{breakurl}
\usepackage[breaklinks]{hyperref}
\def\UrlBreaks{\do\/\do-}
\usepackage{color}
\usepackage{slashed}
\usepackage{multirow}
\usepackage{placeins}
\usepackage{csquotes}
\usepackage{enumitem}
\usepackage[dvipsnames]{xcolor}
\usepackage{epstopdf}
\usepackage{soul}
\usepackage[separate-uncertainty=true]{siunitx}
\usepackage{mathrsfs}
\usepackage{booktabs}
\usepackage{tcolorbox} % Textbox in front of paper chapters

\graphicspath{{img/}}

\usepackage{rotating}
%\usepackage{subcaption} % causes clash with subfigure
\usepackage{adjustbox}
\usepackage{tabularx}
\usepackage{caption}

%  settings
\usepackage{tikz}
\usetikzlibrary{trees}
\usetikzlibrary{decorations.pathmorphing}
\usetikzlibrary{decorations.markings}
\usetikzlibrary{arrows.meta, chains, positioning}
\usetikzlibrary{snakes}
\usetikzlibrary{bending}
\usepackage{pgf-pie}
\usepackage[compat=1.1.0]{tikz-feynman}

\tikzset{% inspired by https://tex.stackexchange.com/a/316050/121799
	arc arrow/.style args={%
		to pos #1 with length #2}{
		decoration={
			markings,
			mark=at position 0 with {\pgfextra{%
					\pgfmathsetmacro{\tmpArrowTime}{#2/(\pgfdecoratedpathlength)}
					\xdef\tmpArrowTime{\tmpArrowTime}}},
			mark=at position {#1-\tmpArrowTime} with {\coordinate(@1);},
			mark=at position {#1-2*\tmpArrowTime/3} with {\coordinate(@2);},
			mark=at position {#1-\tmpArrowTime/3} with {\coordinate(@3);},
			mark=at position {#1} with {\coordinate(@4);
				\draw[-{Triangle[length=#2,bend]}]
				(@1) .. controls (@2) and (@3) .. (@4);},
		},
		postaction=decorate,
	},
	fermion arc arrow/.style={arc arrow=to pos #1 with length 2mm},
	Vertex/.style={fill,circle,inner sep=1.5pt},
	Vertex2/.style={fill,circle,inner sep=0.75pt},
	Vertex3/.style={fill,circle,inner sep=3pt},
	insert vertex/.style={decoration={
			markings,
			mark=at position #1 with {\node[Vertex]{};},
		},
		postaction=decorate},
	insert vertex2/.style={decoration={
			markings,
			mark=at position #1 with {\node[Vertex2]{};},
		},
		postaction=decorate},
	insert vertex3/.style={decoration={
			markings,
			mark=at position #1 with {\node[Vertex3]{};},
		},
		postaction=decorate}
}

\tikzset{
	gaugeboson/.style={decorate, thick, decoration={complete sines,segment length=2mm}}
}

\tikzstyle{Arrow}=[->, draw={rgb,255: red,0; green,84; blue,159}, line width = 0.7mm]

% siunitx settings
\DeclareSIUnit\year{yr}
\DeclareSIUnit\years{yrs}
\DeclareSIUnit\pc{pc}

% Allow fixed size table columns
\usepackage{array}
\newcolumntype{L}[1]{>{\raggedright\let\newline\\\arraybackslash\hspace{0pt}}m{#1}}
\newcolumntype{C}[1]{>{\centering\let\newline\\\arraybackslash\hspace{0pt}}m{#1}}
\newcolumntype{R}[1]{>{\raggedleft\let\newline\\\arraybackslash\hspace{0pt}}m{#1}}

% Allow page breaking in math environments
\allowdisplaybreaks

% Disallow flexible paragraph spacings
\raggedbottom

% No indent after every paragraph
\setlength\parindent{0pt}
\setlength{\parskip}{8pt}

% Macros
\newcommand{\abs}[1]{\ensuremath{\lvert#1\rvert}}
\newcommand{\SU}[1]{\ensuremath{\mathrm{SU}(#1)}}
\newcommand{\U}[1]{\ensuremath{\mathrm{U}(#1)}}
\newcommand{\dd}{\ensuremath{\mathrm{d}}}
\newcommand{\BigO}[1]{\ensuremath{\mathcal{O}\!\left(#1\right)}}

% Variables
\newcommand{\MS}{\ensuremath{\overline{\mathrm{MS}}}}
\newcommand{\hc}{\ensuremath{\mathrm{h.c.}}}
\newcommand{\Charge}{\ensuremath{\mathbb{C}}}
\newcommand{\Parity}{\ensuremath{\mathbb{P}}}
\newcommand{\TeV}{\ensuremath{\text{ TeV}}}
\newcommand{\alphas}{\ensuremath{\alpha_s}}
\newcommand{\muBar}{\ensuremath{\bar{\mu}}}

% Figures
\newcommand{\figurewidth}{0.9\textwidth}

% This essentially duplicates \substack, but adding an alignment point.
\makeatletter
\newcommand{\subalign}[1]{%
  \vcenter{%
    \Let@ \restore@math@cr \default@tag
    \ialign{\hfil$\m@th##$&$\m@th{}##$\hfil\crcr
      #1\crcr
    }%
  }%
}
\makeatother


\newcommand{\todo}[1]{\textcolor{red}{\bf TODO: #1}}

\newcommand{\msout}[1]{\text{\sout{\ensuremath{#1}}}}
\newcommand{\citep}[1]{\cite{#1}}


\definecolor{PlotGreen}{HTML}{3cb44b}
\definecolor{PlotRed}{HTML}{e6194b}
\definecolor{PlotYellow}{RGB}{255, 225, 25}

\usepackage{bm} % bold math letters


% ********************************************************************
% Her Majesty herself
% ********************************************************************
\usepackage{thesis}


% ********************************************************************
% Fine-tune references (hyperref should be called last)
% ********************************************************************
\hypersetup{%
  %draft, % hyperref's draft mode, for printing see below
  colorlinks=true, linktocpage=true, pdfstartpage=3, pdfstartview=FitV,%
  % uncomment the following line if you want to have black links (e.g., for printing)
  %colorlinks=false, linktocpage=false, pdfstartpage=3, pdfstartview=FitV, pdfborder={0 0 0},%
  breaklinks=true, pageanchor=true,%
  pdfpagemode=UseNone, %
  % pdfpagemode=UseOutlines,%
  plainpages=false, bookmarksnumbered, bookmarksopen=true, bookmarksopenlevel=1,%
  hypertexnames=true, pdfhighlight=/O,%nesting=true,%frenchlinks,%
  urlcolor=RWTHviolett, linkcolor=Black, citecolor=RWTHorange, %pagecolor=RoyalBlue,%
  %urlcolor=Black, linkcolor=Black, citecolor=Black, %pagecolor=Black,%
  pdftitle={\myTitle},%
  pdfauthor={\textcopyright\ \myName, \myUni, \myFaculty},%
  pdfsubject={},%
  pdfkeywords={},%
  pdfcreator={pdfLaTeX},%
  pdfproducer={LaTeX with thesis by Tom Schellenberger}%
}


% ********************************************************************
% Setup autoreferences (hyperref and babel)
% ********************************************************************
% There are some issues regarding autorefnames
% http://www.tex.ac.uk/cgi-bin/texfaq2html?label=latexwords
% you have to redefine the macros for the
% language you use, e.g., american, ngerman
% (as chosen when loading babel/AtBeginDocument)
% ********************************************************************
\makeatletter
\@ifpackageloaded{babel}%
  {%
    \addto\extrasamerican{%
      \renewcommand*{\figureautorefname}{Figure}%
      \renewcommand*{\tableautorefname}{Table}%
      \renewcommand*{\partautorefname}{Part}%
      \renewcommand*{\chapterautorefname}{Chapter}%
      \renewcommand*{\sectionautorefname}{Section}%
      \renewcommand*{\subsectionautorefname}{Section}%
      \renewcommand*{\subsubsectionautorefname}{Section}%
    }%
    \addto\extrasngerman{%
      \renewcommand*{\paragraphautorefname}{Absatz}%
      \renewcommand*{\subparagraphautorefname}{Unterabsatz}%
      \renewcommand*{\footnoteautorefname}{Fu\"snote}%
      \renewcommand*{\FancyVerbLineautorefname}{Zeile}%
      \renewcommand*{\theoremautorefname}{Theorem}%
      \renewcommand*{\appendixautorefname}{Anhang}%
      \renewcommand*{\equationautorefname}{Gleichung}%
      \renewcommand*{\itemautorefname}{Punkt}%
    }%
      % Fix to getting autorefs for subfigures right (thanks to Belinda Vogt for changing the definition)
      \providecommand{\subfigureautorefname}{\figureautorefname}%
    }{\relax}

\makeatother


% ********************************************************************
% Development Stuff
% ********************************************************************
\listfiles
%\PassOptionsToPackage{l2tabu,orthodox,abort}{nag}
%  \usepackage{nag}
%\PassOptionsToPackage{warning, all}{onlyamsmath}
%  \usepackage{onlyamsmath}


% ****************************************************************************************************
% 7. Further adjustments (experimental)
% ****************************************************************************************************
% ********************************************************************
% Changing the text area
% ********************************************************************
%\areaset[current]{312pt}{761pt} % 686 (factor 2.2) + 33 head + 42 head \the\footskip
%\setlength{\marginparwidth}{7em}%
%\setlength{\marginparsep}{2em}%

% ********************************************************************
% Using different fonts
% ********************************************************************
%\usepackage[oldstylenums]{kpfonts} % oldstyle notextcomp
%\usepackage[osf]{libertine}
%\usepackage[light,condensed,math]{iwona}
%\renewcommand{\sfdefault}{iwona}
%\usepackage{lmodern} % <-- no osf support :-(
%\usepackage{cfr-lm} %
%\usepackage[urw-garamond]{mathdesign} <-- no osf support :-(
%\usepackage[default,osfigures]{opensans} % scale=0.95
%\usepackage[sfdefault]{FiraSans}
% \usepackage[opticals,mathlf]{MinionPro} % onlytext
\usepackage{FiraSans}
% ********************************************************************
%\usepackage[largesc,osf]{newpxtext}
%\linespread{1.05} % a bit more for Palatino
% Used to fix these:
% https://bitbucket.org/amiede/classicthesis/issues/139/italics-in-pallatino-capitals-chapter
% https://bitbucket.org/amiede/classicthesis/issues/45/problema-testatine-su-classicthesis-style
% ********************************************************************
% ****************************************************************************************************


% Try unappendix
%\makeatletter
%\newcounter{savesection}
%\newcounter{apdxsection}
%\newcounter{savechapter}
%\newcounter{chapterappendix}

%\renewcommand\appendix{\par
%	\setcounter{savesection}{\value{section}}%
%	\setcounter{savechapter}{\value{chapter}}%
%	\setcounter{section}{0}%
%	\setcounter{chapterappendix}{\value{chapter}}%
%	\setcounter{subsection}{0}%
%	\gdef\thesection{\arabic{savechapter}.\@Alph\c@section}%
%}

%\newcommand\unappendix{\par
%	\setcounter{apdxsection}{\value{section}}%
%	\setcounter{section}{\value{savesection}}%
%	\setcounter{chapter}{\value{savechapter}}%
%	\setcounter{subsection}{0}%
%	\gdef\thesection{\thechapter.\@arabic\c@section}%
%}
%\makeatother